\documentclass[12pt]{article}
%
\usepackage[T1]{fontenc}% these specify the font used
\usepackage{garamondx}% default font: Garamond 
\usepackage[garamondx,cmbraces]{newtxmath}
\renewcommand*\ttdefault{cmtt}% specifies typewriter font
\usepackage{cabin}% specifies sans-serif font
%
\usepackage{
  amsmath,% facilitates math formulae typography
  amssymb,% several other symbols
  graphicx,% enables graphics insertion
  color% enables colored text
}
\usepackage[
frak= boondox, scr= boondoxo, cal= cm, bb= boondox
]{mathalfa}% do not alter the order of this line (for strange reasons)
% available ones: (* has small cases)
% bb: pazo, fourier, esstix, *boondox, px, txof.
% frak: *esstix, *boondox, *pxtx
% cal & scr: rsfs, rsfso, pxtx, *esstix, *boondox, *boondoxo, *dutchcal. 
%
% Formatting.
\setlength{\parskip}{1.5ex}% vertical spacing between paragraphs
\setlength{\parindent}{4ex}% indent in a paragraph
\usepackage{titling}% controls typesetting the title
\setlength{\droptitle}{-2cm}% decreases spacing over the title
\usepackage[
  top=2.1cm, bottom=1.9cm, left=1.8cm, right=1.8cm
]{geometry}% sets page margins
\usepackage[compact]{titlesec}% adjusts spacing of each sec.; used below.
\titlespacing{\section}{8ex}{*0}{*0}% resp., left margin, vertical spacing, seperation to text following.
\titlespacing{\subsection}{2ex}{*0}{*2}% see above
\titlespacing{\subsubsection}{0pt}{*0}{*0}
\titleformat{\subsection}[% modifies the title of a subsec.
  runin% no new-line before subsequent text
]{
  \color{red}\large\bfseries\itshape %color, font, shape
}{}{}{}[]% end \titleformat
%\everymath{\displaystyle}% forces displaying in-text math w/ full height
%
% Custom shorthands.
% lower case Greek alphabets w/ long name
\newcommand\aG\alpha \newcommand\bG\beta  \newcommand\gG\gamma \newcommand\dG\delta \newcommand\eG\varepsilon \newcommand\zG\zeta \newcommand\tG\theta \newcommand\kG\kappa \newcommand\lG\lambda \newcommand\sG\sigma \newcommand\fG\varphi \newcommand\oG\omega 
% upper case Greek alphabets
\newcommand\GG\Gamma \newcommand\DG\Delta \newcommand\TG\Theta \newcommand\LG\Lambda  \newcommand\SG\Sigma \newcommand\OG\Omega
%
% other symbols
\newcommand\oo\infty% infinity, whose shape resembles "oo"
\newcommand\F\frac% "F"raction
\newcommand\R\sqrt% "R"oot
\newcommand\M\cdot% "M"ultiply
\newcommand\N\nabla% del sign
\newcommand\X\times% cross, whose shape resembles "X"
\newcommand\Pt\partial% "P"ar"T"ial differentiation
\newcommand\V\boldsymbol% bold italic, e.g. "V"ectors
\newcommand\Ev\forall% "Ev"ery
\newcommand\Ex\exists% "Ex"ists
\newcommand\Mp\mapsto% "M"a"p"
\newcommand\St{\textsf{\large \: s.t. \:}}% "S"uch "T"hat
\newcommand{\Eq}{\Leftrightarrow}% "Eq"ivelent
\newcommand{\Ip}{\Rightarrow} % "I"m"p"lies
\newcommand{\ii}{ \mathring{\imath} }% for imag. unit
\newcommand{\jj}{ \mathring{\jmath} }% for imag. unit
\newcommand{\dd}{ \BF{d} }% for differential
\newcommand{\ee}{ \BF{e} }% for natural base
%
% brackets, customized fonts
\newcommand{\Rb}[1]{ \left( #1 \right) }% "R"ound "b"racket, or more commonly parenthesis
\newcommand{\Sb}[1]{ \left[ #1 \right] }%("S"quare) "b"racket
\newcommand{\Cb}[1]{ \left\{ #1 \right\} }%("C"urly) "b"race
\newcommand{\Ab}[1]{ \left\langle #1 \right\rangle } %Chevrons, e.g. "A"ngle brackt
\newcommand{\Nm}[1]{ \left| #1 \right| } %"N"or"m"
\newcommand{\Bk}[2]{ \left\langle #1 \middle| #2 \right\rangle } %"B"ra-"K"et notation
\newcommand{\Nb}[1]{ \quad \mbox{\color{blue}[#1]} \quad }%"N"ota "b"ene, i.e. note
\newcommand{\Emph}[1]{ {\color{blue}\bfseries{#1}} }% my emphasis
\newcommand{\BF}[1]{ \mathbb{#1} }% "B"lackboard "F"ont
\newcommand{\CF}[1]{ \mathcal{#1} }% "C"ursive "F"ont
\newcommand{\GF}[1]{ \mathfrak{#1} }% "G"othic "F"ont
\newcommand{\SF}[1]{ \mathscr{#1} }% "S"cript "F"ont
\newcommand{\Ss}[1]{\textsf{\textbf{#1}}}% "S"ans-"s"erif
\newcommand{\Tw}[1]{\texttt{\textbf{#1}}}% "t"ype"w"riter font, e.g. quoting codes
%
% miscellaneous
\renewcommand\L\label% "L"a"b"el
\newcommand{\EqG}[1]{ \begin{gather}{#1}\end{gather} }% Eqn. Gather
\newcommand{\EqGo}[1]{ \begin{gather*}{#1}\end{gather*} } % unnumbered
\newcommand{\EqA}[1]{ \begin{align}{#1}\end{align} }% Eqn. Align
\newcommand{\EqAo}[1]{ \begin{align*}{#1}\end{align*} }% unnumbered
\newcommand{\EqAS}[1]{ \begin{align}\begin{split}{#1}\end{split}\end{align} }% numbered only once
\newcommand{\Id}{\hspace{5em}}% "I"n"d"ent, esp. in multi-line formulae
%
%%%%%%%%%%%%%%%%%%%%%%%%%%%%%%%%%%%%%%%%%%%%%%%%%%%%%%%%%%%%
\begin{document}
\title{
 \textit{Honours Introduction to} \\
 \textit{\textbf{\Huge Algebra.}} \\
 Notes, Part 1: \\
 \huge\textsc{Groups}
}
\date{}
\author{}
\maketitle
\allowdisplaybreaks[4]% allows page breaks amid eqn. arrays.

\vspace{-3.3cm} %removes vertical spacing 
\hfill{\itshape lectured by prof. Jing Yu} \par
\hfill{\itshape organized by Tzu-Yu Jeng} \par
\hfill{\itshape Sep. 14 to Oct. 12, 2015} \\
\vspace{-0.1cm} 

textbook: Dummit D.S. \& Foote R.M., Abstract Algebra (3rd ed.). Hoboken, New Jersey: John Wiley \& sons (2004). \par
[Herstein I.N., Topic in Algebra (2nd ed.). Hoboken, New Jersey: John Wiley \& sons (2004).] \par
[Fraleigh J.B., A first Course in Abstract Algebra (7th ed.). Boston, Massachusetts: Addison-Wesley (2003).]

\section{hello everyone} 
\subsection*{1.} \Ss{Algebra} concerns objects such as numbers, polynomials, matrices, functions, sequences, all introduced in terms of \Ss{variables}.

\subsection*{2.} This science  originates with the book ``Algebra'' Persian mathematician \Ss{al-Khwarizmi} wrote (ca. 830 CE). 
His name if latinized, gives the etymology of ``Algorithm''!

\section{composition laws} 
\subsection*{3.} A \Ss{composition law}, or \Ss{binary operation}, is a map \(G \X G \Mp G\), which maps \((a, b)\) to a new number we denote as \(a \M b\).

\subsection*{4. examples.} Fix \(n>0\), the set of \(n \X n\) matrices \(a_{ij} \in \BF Z\). 
Multiplication reads \( (AB)(i,j) = \sum_l a_{il} b_{lj} \). 

\subsection*{5.} Consider \Ss{permutations} of the finites set \(\OG := \{1,2,\dotsc,n\}\). 
The composition law is given by just the composition of those functions: \( \sG \M \rho \Mp \sG \circ \rho \).

\subsection*{6.} The operation: \(P_1  P_2 = P_4\), where \(P3\) is the 3rd point of the secant line of \(P_1, P_2\) to the \Ss{elliptic curve} \EqGo{
 y^2 = x^3 - n^2 x,\quad x,y \in \BF{R}
} and \(P_4\) is its mirror w.r.t. \(x\)-axis. \\

\subsection*{7.} With an \Ss{equivalence relation} \(a \sim b\) given on \(S\), 
equivalent classes are thus defined: \( a,b \in \CF{C}, \Ip a \sim b \). 
They are mutually exclusive, and collectively exhaustive: \(S = \cup_g \CF{C}(g),\; \CF{C}(g) \cap C(g') = \varnothing\).

\subsection*{8.} For example, \( a \sim b \Eq n|(a-b) \) gives \Ss{partition} \( \CF{C}(a) = \{a+kn: k \in \BF Z \} \). 

\subsection*{9.} We are inclined to define \( \CF{C}(a) + \CF{C}(b) := \CF{C}(a+b) \) and that \( \CF{C}(a) \M \CF{C}(b) := \CF{C}(ab) \), 
but every time this is done it must be checked that these are well-defined: \begin{gather*}
 a' \sim a \land b' \sim b \Ip 
 \begin{cases}
  a' + b' \sim a+b, \\
  a' \M b' \sim a \M b. 
 \end{cases}
\end{gather*}

\section{groups}
\subsection*{10. def.} A set \(G\), together with a composition law \(\M\), is called a \Ss{group}, if it be the case that, \(\Ev a,b,c \in G\):
\indent (i) \Ss{associativity}:  \( (a \M b) \M c = a \M (b\M c) \) \\
\indent (ii) \Ss{existence of identity}: \( \Ex e \in G, \St e\M a = a\M e = a\) \\
\indent (iii) \Ss{existence of inversion}: \( \Ex a^{-1} \in G, \St a\M a^{-1} = a^{-1} \M a = e \).

\subsection*{11. prop.} The identity \(e\) is unique. \par
\Ss{Proof.} Let \(e, e'\) be both identities. 
Then \(e = ee' = e'\), 
since resp., \(e'\) is identity, and \(e\) is identity.

\subsection*{12. prop.} \(\Ev a \in G\), the inverse of \(a\) is unique. \\
\indent \Ss{Proof.} Let \(b, b'\) be both inverses of a. 
Then \(b' = b'e = b'(ab) = (b'a)b = eb = b\), 
since resp., the nature of \(e\), that \(a,b\) are inverse of each other, that too are \(a,b'\), and the nature of \(e'\). 

\subsection*{13. remarks.} \( (a^{-1})^{-1} = a \). It's left to the reader.
\( (ab)^{-1} = b^{-1}a^{-1} \). It's left to the reader.

\subsection*{14.} \( a_1 a_2 \M \dotsb \M a_n \) is independent of how it is bracketed.  It's left to the reader. [Use math induction.]

\subsection*{15. def.} \(G\) is said \Ss{abelian}, or \Ss{commutative}, when \( ab = ba,\; \Ev a,b \in G \). \par
A \Ss{finite group} \(G\) has finite cardinality \(|G|\). \par
\Ss{Convention}: When \(G\) is abelian, denote as \(a+b\) the operation in question, and denote as 0 the identity. 
We also write \(na := a+a+\dotsb+a\), repeated \(n\) times; 
\(a^n := a \M a \M\dotsb\M a\), repeated \(n\) times; 
where \(n=1,2,3,\dotsc\). 

\subsection*{16. examples.} \( \BF{Z, Q, R, C} \) under addition, are groups. \par
Usual vectors in \( \BF R^n \) under addition, is a group. \par
Matrices with nonzero determinant, \(\CF {GL}(\BF R) = \{ A: A_{ij} \in \BF R \}\), are groups. \par
Permutation \( S_n \) under composition operation, is a group.

\subsection*{17. def.} When \(H \subseteq G\), \(H\), together with the operation \(\M\) of \(G\), is said to make a \Ss{sbgp}, if \(H\) so-considered is a group. 

\subsection*{18. cancellation law.} [Sep. 17] Let \(G\) be a group. Then \[ \begin{cases}
au=av \Ip u=v, \\
ub=vb \Ip u=v
\end{cases} \] for all \(a,b,u,v \in G\). \par
So, equation such as \(ax=b\) or \(ya=b\) has unique solutions. 

\Ss{Proof.} \( au=av \Ip a^{-1}au = a^{-1}av \Ip u=v \). 
The other one is similar. 

\subsection*{19. examples} Trivial group \(G={e}\) is the example in which \(|G|=1\). 
A group of size 2, \(G= \{ p,n \}\) where \EqGo{
 pn=np=n, \quad pp=nn=p,
} may represent positive (\(p\)) or negative (\(n\)) under usual multiplication, or even (\(p\)) and odd (\(n\)) class of natural numbers under addition. 

\subsection*{20.} we have just seen an example of \Ss{isomorphism}. 
In fact, all groups of size 2 is isomorphic. 
Why is that? Let's call the identity \(p\), and we know \(pn = np = n, pp =p\). 
Also, \(nn\) must not be \(n\), because \(n \neq p\). So \(nn=p\). 

\subsection*{21.} Let's call \Ss{finite cyclic group}, and denote as \(\BF Z /p \BF Z\), the group which is isomorphic to mod \(p\) classes. 
In other words, \(0+1=1, 1+1=2, 2+1=3,\dotsc,p+1=0\) and so on. 

\subsection*{22.} \(G_1 := \BF Z \X \BF Z = \{p,n\} \X \{p,n\}\) and \(G_2 := \BF Z /4 \BF Z = \{0,1,2,3\}\) are not isomorphic. \par
To see this, examine any \(h: G_2 \Mp G_1\). 
But \( h(1 + 1) = h(2) = h(1) \M h(1) = p \), the identity, while of course, \(h(0) = p\) also. 
So \(h\) is not an isomorphism. \par
We denote \(|g|\), called the \Ss{order} of \(g\), to be the least number \(k\) so that \(g^k=1\). 

\subsection*{23. classes.} Introduce, for \(g,h \in G\), such relation: \( g \sim h \Eq gh^{-1} \in H \). 
It's easy to see that \(g \sim g\); 
that \(g \sim h \Eq h \sim g\) because \(hg^{-1}\) is just \((gh^{-1})^{-1}\); 
and that \(g_1 \sim g_2 \land g_2 \sim g_3 \Ip g_1 \sim g_3 \). 
So it's an equivalence relation. 

\subsection*{24.} Introduce the notation \(gH :=\) the equivalence class containing \(g\), 
which is called \Ss{left coset} of \(H\). 
In other words \( g_1 H = g_2 H \Eq g_1 \sim g_2 \Eq g_1 g_2^{-1} \in H\). 

\subsection*{25.} More examples of Isomorphism. The exp or log maps \(\BF R\) under +, to \((0,\oo)\) under \(\M\), and log does conversely. 
And \(\Tw{hom}(V,V)\), the set of all linear transformations from \(V\) to \(V\), is \(\cong M_n\), the set of all \(n \X n\) matrices. 

\subsection*{26.} [Sep. 21] We state the definition of a \Ss{ring} at this stage for reference. 
A \Ss{ring} \(R\) has on it two groups, w.r.t. resp. addition ``+'' and multiplication ``\(\M\)'', 
so that \((R, +)\) is an abelian group, while \((R, \M)\), while also contains a identity and subject to association law, is not required to be commutative; 
moreover, there is the distribution law \( c(a+b) = (a+b)c = ac + bc \), and there is a multiplicative identity 1, which \(\neq 0\). 

\subsection*{27.} The \Ss{quaternions} \(Q\) form a ring. 
A \(q \in Q\) is pair \(a +b\SF{i} + c\SF{j} + d\SF{k}\), where \(a,b,c,d \in \BF R\), 
under usual addition as in 4-space, 
and that \EqGo{
 \SF{i}^2\ = \SF{j}^2 = \SF{i}^2 =-1, \\
 \SF{ij} = -\SF{ji} = \SF{k}, \\
 \SF{jk} = -\SF{kj} = \SF{i}, \\
 \SF{ki} = -\SF{ik} = \SF{j}.
}

\subsection*{28.} Choose some elements in \(R\), which has multiplicative inverse, and the collection \(R^\X\) is a ring. For example, \(\{1,-1\} \subseteq \BF Z\) under usual \(+\) and \(\M\). \par
Also, when \(F\) is a field, \(F^\X = F -\{0\}\) is a group under multiplication. \par
\((\BF Z/N\BF Z)^\X\) is a ring under usual modular addition, and modular multiplication. 
The inverse elements are found by the familiar Euclid algorithm. 

\subsection*{29.} Denote \(Q= \{\pm 1, \pm \SF{i}, \pm \SF{j}, \pm \SF{k}\}\), the quaternion group, which is non-abelian. 
It is claimed that any non-abelian group \(G \St |G| = 8\), is isomorphic to either \(D_8\), the dihedral group, or to \(Q\). It's left to the reader to verify. 

\subsection*{30.} Let us turn to permutations \(S_n\). Define \[
\DG := \prod_{1 \leq i < j \leq n} (x_i - x_j), 
\] where \(x_i\)'s are all independent variables. 
And define \[ 
\sG(\DG) := \prod_{1 \leq i < j \leq n} ( x_{\sG(i)} - x_{\sG(j)} ).
\] It's left to the reader to verify that \( \eG(\tau \sG) = \eG(\tau) \eG(\sG)\). 

\subsection*{31.} It's easy to see \(S_1 \cong \{e\}, S_2 \cong \BF Z_2, S_3 \cong D_6\). \par
Notice \(\BF Z_3 \leq S_3\), and that \(\BF Z_4 \leq S_4\). 
Their orders are resp. \(|S_3| = 6, |S_4| = 24, |A_4| = 12\). \par
In general, we have the \Ss{Sylow's thm.}: 
If \(p^l\) divides \(|G|\), and is the highest of such powers. 
Then \(G\) has a sbgp.\ of order \(p^l\). (more on this later!)

\section{action, classes, counting}
\subsection*{32. group action.} A \Ss{group action} is a map \(G \X A \Mp A\). 
\(S_{|A|}\) the symmetric group of \(A\), acts on \(A= \{1,2,\dotsc,n\}\). 
The \(D_{2n}\) rotates or mirrors the set of n-gons. 
Thus motivated, we define the \{group action\} to be a map \(\rho: G \X A \Mp A\), which does something on the group. 

\subsection*{33.} [Sep. 24] \(Gs := {gs: g \in G}\) is called the \Ss{orbit} of \(s\). 
\(G_s := {g: gs =s, g \in G}\) is called \Ss{stabilizer}. \par
\(G\) is partitioned into disjoint orbits. 
Why is that? Denote \(s_2 \sim s_1\) if \(s_2 = gs_1\) for some \(g \in G\). 
it's easy to verify that it's an equivalence relation, 
thus it must have partitioned \(G\) into mutually exclusive classes. 

\subsection*{34. Lagrange's thm.} Consider groups \(H \leq G\), with \(|G|<\oo\). Then \(|G| = |H| |G/H|\). 
Denote by \(G/H\) the collection of all \Ss{left cosets} of \(H\) inside \(G\). 
We just saw that \(gH\)'s are disjoint and (clearly) of the same order, and the result follows. 

\subsection*{35.} Multiplication by some \(g \in G\) gives, we see, another \(gH\). 
\(G\) may be viewed to have acted thus: \(G \X G/H \Mp G/H\). 
Let us view the equation \(|G| = |H| |G/H|\) in another way. 
Because the stabilizer of \(H\) is just \(H\), 
and that, for fixed \(gH\), its orbit is all of \(G/H\). So, \EqGo{
 |G| = |\textrm{stabilizers of}\; H| \M |\textrm{orbits of}\; gH|
}

\subsection*{36.} Let us, then, generalize this for any group \(P\) that acts on any set \(X\): \[
|P| = |P_x| \M |Px|. 
\] This is intuitive by just counting. 
[Fraleigh 7e p.158.] Why is that? Suppose \(p\) carries \(x\) to be \(px\). 
Then \(p\) may be broken into two parts, namely one that does not change \(x\), and another one that carries \(x\) to what it is, that is \(px\). 
The product of two sets makes up the set of all actions, \(|p|\), just the way \(G\) is broken, by \(H\), into cosets \(gH\)'s. 

\subsection*{37. centers, conjugation classes.} Another example is that: consider G that acts on itself [See also Fraleigh 7e p. 328]. 
Here the action in question is the \Ss{conjugation} action, \(G \X G \Mp G\). 
Its order is just \(|G|\). 
That is to say, \((g,x) \Mp gxg^{-1}\). 
Notice how orbits of a certain \(x\), called the conjugation class of \(x\) and denoted by \(:= \Tw{conj}_G (x)\), takes the form \(\{x: gxg^-1, g \in G\}\). 
They are (clearly) sbgp.s. 
As such, \(G\) is split into many conjugation. \par
Consider one of them, \(\Tw{conj}_G (x_i)\). What are those stabilizers? Well, they are such elements that \(gxg^{-1} = x\), or such one as commutes with x, and are called \Ss{centralizers}. 
Every coset, \(g \M \Tw{conj}_G (x_i)\), has order \(|C_i| / |\Tw{conj}_G (x_i)|\). 

\subsection*{38.} Now again, number of all conjugation actions \[
|G| = \sum_{i} \Tw{conj}_G (x_i)
\] where sum is taken through all distinct conjugation classes, \(C_i\). 
Let's further define the \Ss{center} of \(G\) to be \(Z_G := \{z: zg = gz,\; \Ev g \} \). 
Notice \(Z \leq C_G(x),\; \Ev x \in G\). 
Together with above observations, we rewrite the above explicitly to be: \EqGo{
 |G|
 = |Z_G| + \sum_{i} |\Tw{conj}_G (x_i)|
 = |Z_G| + \sum_{i} \F{|G|}{|C_G(x_i)|}
} and sum is taken through all non-trivial conjugation classes. 

\section{Sylow theory}
\subsection*{39. def.} \(G\) is said a \Ss{\(p\)-group}, if \( |G| = p^k \) for some prime, \(p\). \par

\subsection*{40. first part.} Let any finite group \(G\), and prime \(p\) that divides \(|G|\), be given, and \(|G| = mp^k\), \(p\) not dividing \(m\). 
Then there exists at least a sbgp.\ \(H\) which \(\leq G\), with \(|H| = p^k\). \par
\(H\) is called a \Ss{Sylow-\(-\)-sbgp}. Call the set \(\Tw{syl}_p(G)\) which is the collection of all \(p\)-sbgp.s, and suppose there are \(s:=|\Tw{syl}_p(G)|\) such ones. 
(``Sylow'' is pronounced, roughly in IPA, ``sylov''; ``y'' as in German ``\"u''.) \par

\subsection*{41. second part.} All Sylow \(p\)-sbgp.s are conjugate to each other. 
That is to say, \(\Ex g \St g H g^{-1} = J,\; \Ev H, J \in \Tw{syl}_p(G)\). \par
Moreover, \(\Ev K\) which is \(\leq G\) and is of order \(|K| = p^i\), 
we have \(K \leq\) some \(H \in \Tw{syl}_p(G)\). \par

\subsection*{42. third part.} Not only does \(s\) divide \(m\), but \(s =1\) (mod \(p\)). 
In particular, if \(s =1\), so that \(H\) is the only Sylow \(p\)-sbgp.\ of \(G\), 
it follows that \(H\) must be normal in \(G\). \par
We shall present the proof the Sylow theorems by largely using only counting technique, which we shall see to be powerful. 

\subsection*{43. corollary.} Still let \(G\) be a \(p\)-group. 
Then \(Z_G \neq {e}\), that is \(G\) contains a non-trivial sbgp. \par
\Ss{Proof.} [Fraleigh 7e p.329] Refer to the class equation. 
All \(|\Ss{conj}_G (x)|\) are non-trivial, and, for they are sbgp.s, divisible by \(p\). 
Thus \(|Z_G|\) must have been divisible by \(p\) also.

\subsection*{44. corollary.} if \(|G| = p^2\), then \(G\) is abelian. \par
\Ss{Proof.} [Fraleigh 7e p.329] Pick up sbgp.\ \(A\) and \(B\), each of order \(p\), as requested by Sylow 1st thm. 
They are cyclic, because of Lagrange's thm. 
Moreover they are normal, as Sylow 3rd thm. said. 
This enables we write \(G\) as the product \(\Ab{a} \X \Ab{b}\), where \(a, b\) are resp. their generator. \par
But does \(a^h b^k\) = \(b^k a^h\)? 
Consider any \((a^h b^k a^{-h}) b^{-k}\), which is \(\in B\), a normal group of \(G\). 
But, if written \(a^h (b^k a^{-h} b^{-k})\), it's similarly in \(A\). 
If \(A \neq B\), the only possibility is that \(a^h b^k a^{-h} b^{-k} = 1\). 

\subsection*{45. lemma (a).} Let nonempty \( V \subseteq G \). 
Here \(G\) is considered to have acted on \(V\) by right-multiplication \(gV\). 
Then \(G_V\), the stabilizer of \(V\) has order \(|G_V|\) that must divide \(|V|\). \par
\Ss{Proof.} \( G_V = \{g: g\in G, gV = V\} \) by definition. 
A coset \(gV\) consists of elements of \(V\). 
However, we may well say \(G_V\) (being a group) produces cosets by being right multiplied by \(V\), 
and \(V\) is partitioned into \(|G_V|\) orbits. 
So \(|G_V|\) divides \(|V|\). \par

\subsection*{46. lemma (b).} Binomial coefficient \( M := \CF C_{p^l}^n \) is not divisible by \(p\). (Recall \(n=|G|=mp^l\)) \par
\Ss{Proof.} Just expand \EqGo{
 M = \F{ n(n-1)\dotsb(n-p^l+1) }{ p^l(p^l-1)\dotsb1 },
} and consider some \(k= p^i l\), where \(i<l\), and \(p\) doesn't divide \(l\). 
Notice if such \(k\) is met in some factor in the numerator, \(n-k = p^l m - p^i l = p^i (p^{l-i} m - l)\), 
while in the denominator \(p^l - k = p^i (p^{l-i} - l)\). 
So the \(p^i\) factors just exactly cancel! 

\subsection*{47. proof of the first part.} Let \(\SF S\) be the collection of such subsets of \(|G|\), as with cardinality \(p^l\), i.e., \(\SF S := \{ V: |V|=p^l, V \subseteq G\}\). 
Consider \(G\) to have acted on \(\SF S\). 
This way, \(\SF S\) is divided into orbits. 
Since, as what we've saw in lemma (b) (item 46), \(p\) doesn't divide \(N\), we may well choose some \(V \in S\), whose orbit has cardinality not divisible by \(p\). 
(otherwise \(p\) would have divided \(|\SF S|\)) 
But \(|G| = mp^l = |G_V| |GV|\) (see item 35). 
Moreover by lemma (a) (item 45), \(|G_V|\) has factor that' power at most \(p^l\). 
Hence \(|G_v|\) is not only power of \(p\), but, in fact, multiple of \(p^l\). 

\subsection*{48. proof of the second part.} Same settings. 
Let \(H\) be a Sylow \(p\)-sbgp, here fixed. 
Let \(\SF H\) be the collection of \(gH\), namely left cosets of \(H\). 
\(G\) acts on \(\SF H\) by left multiplication. \par
Note \(G_{H}\), the stabilizer of the coset \(gH\), is \(H\) itself. 
Why? Certain \(g \in G_{H}\), which sends \(\Ev h_1\) to another \(h_2\), would be in \(H\); 
also, clearly, \(H \subseteq G_{H}\). 
So \(K_{H} =H\). 
Then \(|\SF H| = mp^l/ p^l =m\) (see item 35), and \(p\) doesn't divide \(m\). \par
On the other hand, pick \(K\) which is any p-sbgp.\ of G, and now consider it also to have acted on cosets \(gH\)'s. 
Then \(\SF H\) is the sum of orders of disjoint orbits, \(|KgH|\)'s. 
But \(p^l = |K| = |K_{gH}| \M |KgH|\), 
therefore \(p\) divides every non-trivial \(|KgH|\). 
But we just saw \(|\SF H|\) would not divide \(m\). 
How can it be? It must be the case that at least one \(|KgH|\) is trivial, and \(|K_{gH}|\) is all of \(K\). \par
In other words, \(kgH \subseteq gH \Ip kgh = gh',\; \Ev h\). 
That is to say, \(k= gh'h^{-1}g^{-1} \in gHg^{-1}\). 
So we found that \(K_{gH} \leq gHg^{-1}\), which is also a \(p\)-group (since its order is preserved), as was asserted. \par
If moreover, \(J\) is another Sylow \(p\)-sbgp, 
the very same argument forces \(J\) to be all of \(gHg^{-1}\). 

\subsection*{49. example.} We investigate the case of \(S_3\), and set \(p=\) 2 and 3, resp. 
\(|S_3|=6\), so there are three Sylow-2-sbgp.s, namely ones formed by the identity and, resp., \((1,2),\; (1,3),\; (2,3)\). 
And one Sylow-3-sbgp, namely one that generated by \((1,2,3)\). 

\subsection*{50. def.} \(G\) and \(H\) are groups, and \(H\), which \(\leq G\) is said to be a \Ss{normal sbgp}, if \(gHg^{-1} = H,\; \Ev g \in G\). This, we denote by \(H \lhd G\).

\subsection*{51. proof of the third part.} Consider the conjugation action acted on \(G\), by \(G\) itself. 
The \Ss{normalizer}, \(|N_G(H)|\) (the set in which \(H\) is normal), being (clearly) a sbgp, has order dividing \(|G|\). 
Again let \(H \in \Tw{syl}_p(G)\), 
and recall \(m= |G|/|H|\). 
By 2nd part, all \(H' \in \Ss{Syl}_p(G)\) are conjugate to \(H\), and so is, in other words, the orbit of \(H\) by conjugation. 
We conclude \(s= |\Tw{syl}_p(G)|= |G| / |N_G(H)|\) (see item 35). 
These relations indicates \(s\) divides \(m\). \par
Fix \(H \in \Tw{syl}_p(G)\), that now acts on \(\Tw{syl}_p(G)\) by conjugation, that is to send \(J\) to \(hJh^{-1}\). 
We claim that the orbit of \(H\) is just \(H\), 
and that other orbits of \(H\) has cardinality being multiple of \(p\). 
If these be the case, then \(s = 1 + lp = 1\) (mod \(p\)). \par
The former is easy, for if \(hjh^{-1} \in H, h \in H\), it's seen that \(j \in H\); also, conjugation is injective. 
As of the latter, we investigate whether \(H_K\), where also \(H \neq K \in \Tw{syl}_p(G)\), can have been all of \(H\). 
By def., \(h H_K h^{-1} \subseteq H_K\); 
but by Sylow's 2nd thm., \(h H_K h^{-1}\) also \(\subseteq H\). 
Because conjugation action is injective, it's implied that \(H=K\). \par
So we just choose some \(K \neq H\), 
so that \(H_K\) cannot have been \(H\). 
Together with the fact that \(|H|= |H_K| \M |HK|\), 
we know \(|HK|\) is not unity and contain factor \(p\), 
and the proof is finally complete. \par
Remark that when \(n_p\) is just 1, conjugation carries \(H\) onto itself, or \(H \lhd G\). 

\subsection*{52.} [Oct. 1] We have already encounters families of group of many sorts, for instance cyclic groups \(Z_n, n \geq 1\), 
dihedral group \(D_{2n}, n \geq 3\), symmetric groups \(S_n, n \geq 1\). \par
Inside a symmetric group, \( \sG, \tG \in S_n \), what does conjugation by \( \tG: \tG \sG \tG^{-1} \in S_n \) look like?
After \(\tG\sG\), it becomes \( (\tG(a_1) \dotsb \tG(a_1)) (\tG(b_1) \dotsb \tG(b_1)) \) and so forth.
Last, \(\tG^{-1}\) just operates on elements, and observe the length of cycles does not change. \par
In conclusion, two elements in \(S_n\) are conjugate, if and only if they have the same type of cycle partition length. \par

\subsection*{53. corollary.} For \(n \geq 3\), \(Z(S_n)\), center of \(S_n\), must have been merely trivial and \(=\{1\}\). 
proof: Choose some element \(\sG\), which we may well call \((1,\dotsc,n)\). 
Just take conjugate of \(\tG=\) (1,(n+1)): clearly \(\tG \sG \tG^-1\) mixed two cycles and exhibit different partition. 
We see \(\sG\) just cannot have commuted with all elements. 
Such examples can always be found. \par
Notice that just by observing \(\SF S_n\) we have found many things!

\section{isomorphism}
\subsection*{54. def.} For homomorphism \( f: G_1  \Mp G_2 \), 
we call \(\Tw{ker}(f) := \{ g: g \in G_1 , f(g) =1 \}\). \par
Remark that: \(\Tw{img}(f) = f(G_1) \leq G\). \par
And \(\Tw{ker}(f) \leq G_1\), because \(f(g_1 g_1') = f(g_1)f(g_1') = 1\). 
And in fact \(\Tw{ker}(f) \lhd G_1\), because \(f(g_1 h g_1^{-1}) = f(g_1) h f(g_1^{-1}) = f(g_1 g_1^{-1}) = 1 \).

\subsection*{55. thm.} Consider \(N \leq G\). These are equivalent:
(i) \(N \lhd G\). 
(ii) \(N_G(N) = G\). 
(iii) \(gN = Ng,\; \Ev g \in G\). 
(iv) \(G/N\) has a natural group structure, namely \(g_1 N g_2 N = g_1 g_2 N\).
(v) \(gNg^{-1} \leq N\) \par
\Ss{Proof.} They are all straightforward. 

\subsection*{56.} \Ss{1st Isomorphism theorem.} (due to E. N\"other) Let \(\fG: G \Mp H\) be a bijection, with \(K:= \Tw{ker}(\fG)\). 
Then \(G / K \cong H\). \par
\Ss{Proof.} Define \(\psi(g K) = \fG(g)\). 
If \(\fG(g_1) = \fG(g_2)\), 
then \(\fG(g_1)^{-1} \fG(g2) = \fG(g_1^{-1} g_2) = 1 \), 
and that \(g_2 = g_1 k\) for some \(k\). 
Thus, well defined. \par
Remark \(\{1\} \leq G\), and \(G \leq G\) (trivial sbgp.s).

\subsection*{57. def.} a group \(G\) is said \Ss{simple} if it possesses no non-proper normal sbgp. 
We will see later that alternating groups \(A_n, n \geq 5\) are simple. 
And that all group with prime numbers are cyclic, and thus simple. \par
It is a difficult task, called \Ss{H\"older Program}, to get \(G\) from known normal \(N\) and \(G/N\). 

\subsection*{58. examples.} Every group action is isomorphic to some sbgp.\ of symmetric \(S_n\). \par
Consider \(S_3\), which has 3 Sylow 2-sbgp.s, namely \(A := \{(12), (13), (23)\}\). 
\(G= S_3\) itself acts on \(A\) via conjugation. \par
As another example, \(D_2n\) acts on regular \(n\)-gon. 
\(D_6 \cong S_3\), while in general, \(D_2n \Mp\) something \(\leq S_n\). 

\subsection*{59.} Let us investigate example: \(|G| = 12\). How many Sylow 3-sbgp.\ are there? either only 1, or 4 which is = 1 (mod 3). 
Now suppose G has 4 mutually-conjugated Sylow 3 sbgp.s. 
We claim: \(G \cong A_4\), where \(|A_4| = 12\) is the alternating group (sbgp.\ of \(S_4\) that can be written as even alternations).

Why is that? \(G \X \Tw{syl}_3 \Mp \Tw{syl}_3\) by conjugation. 
Thus considered as permutation, \(G \Mp\) something \(\leq S_4\), c.t. four Sylow groups. 
Inside \(G\) there are 8 elements of order 3, namely those in 4 order-3 Sylow groups other than 1. 
Note \(|f(G)| > 8\), and \(|f(G)| = 12\) that divides 24. 
The only possibility of such order is \(A_4\). 

\subsection*{60. examples.}  \(p\) being prime, \(\neq 2\). 
\(G := S_{2p}\), the symmetric group. 
Let us choose \(p\) elements among \(A\) on which \(S\) acts. 
There are \(\SF C_p^{2p})\) choices. 
Clearly such \(\sG\) is of order \(p\). 
and the other \(p\) elements permutes among themselves also. 
This sbgp.\ c.t. such choice \(\cong Z_p \X Z_p\). 

\subsection*{61. thm.} Suppose both \(H, K \lhd G\), and \(H \cap K = \{1\}\). 
Then  \(G \geq HK \cong H \X K\). \par
\Ss{Proof.} Closure follows easily from \(H, K\) being normal. 
Now every element in \(HK\) can be written uniquely as \(hk\). 
Indeed, \(h_1 k_1 = h_2 k_2 \Ip h_2 h_1^{-1} = k_1 k_2^{-1} = 1\) by assumption, so \(h_1 = h_2, k_1 = k_2\). 

\subsection*{62. corollary.} Given \(|G| < \oo\), with distinct prime divisor, and suppose that there is only 1 Sylow-\(p\) sbgp.\ for every \(p\) being prime that's a factor of \(|G|\). 
Then, by above, and that all Sylow groups are normal in \(G\), their direct product make up a group, and is all of \(G\) --- there certainly is no other possibility of their order. 

\subsection*{63.} Suppose \(H,K \leq G\), while \(H \lhd G\), but K not necessarily normal, and \(H \cap K = 1\). 
Observe: \(HK := \{ hk: h\in H, k\in K \} \leq G\), 
because, as above, closeness follows: \(h_1 k_1 h_2 k_2 = k_1 k_1^{-1} h_1 k_1 h_2 k_2 = k_1 h_1' h_2 k_2\)

\section{automorphism, semidirect product}
\subsection*{64. def.} The collection said a \Ss{automorphism} is defined \(\Tw{aut}(G) := \{f: G \Mp G\}\), where \(f\)'s are isomorphisms. 
Its product is just composition of function. 
Clearly it's a group. 
It is (loosely put) the symmetric group that manipulates the group in question. 

\subsection*{65.} For example, \(\BF Z / n \BF Z \cong \BF Z_n\). 
\(\Tw{aut}(\BF Z_n) \cong (\BF Z / n \BF Z)^\X\).
In this regard \(|\Tw{aut}(\BF Z_n)| \cong \fG(n)\), where \(\fG\) is Euler totient function. 
A difficult problem that interests many is that, given particular group \(G\), to compute \(\Tw{aut}(G)\). \par

\subsection*{66. def.} Let a function \(\fG: k \Mp \Tw{aut}(G)\) be known. 
A \Ss{semi-direct product} sends elements in the set formed by cartesian product \(H \X K\) to the new set --- which now is a group --- denoted as \(G := H \rtimes K\), 
in such a manner: \((h_1, k_1) \M (h_2, k_2) := (h_1 (\fG(k_1))(h_2), k_1 k_2)\). \par

\subsection*{67. remarks.} If \(k\) is mapped to conjugation, then \((h_1, k_1) \M (h_2, k_2) := (h_1, k_1 h_2 k_1^{-1}, k_1 k_2)\). \par
If \(khk^{-1} = h\), that is \(K\) and \(H\) commute, the semi direct product is just direct product, for the product reduces to \((h_1 h_2, k_1 k_2)\). 
It has inverse element \((h,k)^-1 \in H \rtimes K\) which is \(\fG(k^{-1})(h^{-1}), k^{-1})\), as readily verified. 

\subsection*{68. examples.} Notice \(D_{2n} \cong (Z_n \rtimes Z_2)\) where the mapping rule is \(\fG(s): r^m \Mp r^{-m} \in \Tw{aut}(Z_n)\), as can be verified. 
Thus, \(\BF Z_n \lhd D_{2n}\) (easy to verify this in general). 
One of the other possibility, like \((\fG(s))(r) = r\), makes up a direct product. 

\subsection*{69. examples.} Another example: the \Ss{infinite dihedral group}, on the other hand, is formed in this manner: \(\BF Z \rtimes \BF Z_2 := D_\oo\). 
Notice \(\BF Z\) may be considered here as an infinite cyclic group. 

\subsection*{70.} [Oct. 5] Recall that group action on \(G\) acts thereon by left multiplication. 
Consider finite \(|G|\). Thought reveals that homomorphism \(G \Mp S_{|G|}\), and moreover is 1-1. 
That is, G is isomorphic to a sbgp.\ of \(S_n\). \par
Note there is a natural injective homomorphism \(S_n \Mp \CF{GL}(n,F)\), the collection of all invertible \(n \X n\) matrix. 
Why so? Take \(F^n\) as vector space over \(F\), and take a basis of \(F^n\), then elements in \(\CF{GL}(n,F)\) invertible linear transform \(F^n\) to \(F^n\). 
Then permutation is represented by permutation matrices over \(F\). 
Such \Ss{linear representations} will be studied in some detail in later of the course. 

\subsection*{71.} We investigate: if \(H \leq G\), and \(G = H \cup aH\) are disjoint, i.e., with \(a \notin H\). 
Now \([G:H] =2\) (so 2 divides \(|G|\)), 
hence \(\Ev g \in G, gHg^{-1}\) must be back into \(H\). In general, we have the result ---

\subsection*{72. thm.} Given group \(G\), let \(p\) be the smallest prime dividing \(|G|\), then any sbgp.\ \(H\) of index \(p\) must be normal in \(G\). \par
\Ss{Proof.} \(G\) is considered to have acted on left cosets of \(H\) according to left multiplication: \(G \X (G/H) \Mp (G/H)\). 
Naturally, there is a correspondence \(\pi_H: G \Mp S_{G:H} = S_p\). \par
\(\Tw{ker}(\pi_H)\) is \(\in H\). Why so? If when multiplied by some \(k\), no coset \(gH\) shifts. 
Then every \(h\) is sent to another one within \(H\), 
and then \(k \in H\). 
Suppose \([H: \Tw{ker}(\pi_H)] =l\). 
Counting gives \(|G: \Tw{ker}(\pi_H)| = |G:H| \M |H: \Tw{ker}(\pi_H)| := pl\). \par
However, we just saw all non-identity actions is \(<\) all permutations. 
By 1st isomorphism thm., \(G / \Tw{ker}(\pi_H) \Mp S_p\). 
\(\Ip pl | p! \Ip l | (p-1)!\). But, by above, \(l\) is a divisor of \(|G|\), and every prime factor, if any, of \(l\) divides \(G\) also. Therefore \(l=1\). 
Now that \(H\) is a kernel of homomorphism, we know it's normal. 

\subsection*{73. def.} For \(A < G\), the smallest sbgp.\ of \(G\) containing \(A\) is denoted by \(\Ab{A} = \cap \{H: A \leq H \leq G \}\). 
In other words, this is the sbgp.\ generated by \(A \leq G\). 
If \(A={x}\), then \(\Ab{A} =\) cyclic sbgp.\ generated by \(A\). 

\subsection*{74.} Application: let's compute centralizers in \(D_8\). 
Recall it's defined as \(\CF C_{D_8}(s) = \{g \in D_8: gs = sg\} \leq D_8\). 
It's seen: \(r^2 \in \CF C_{D_8}(s), r \notin \CF C_{D_8}(s)\), and that \(s \in \CF C_{D_8}\). \(\Ip \Ab{s,r^2} \leq \CF C_{D_8}(s) \Ip \Ab{s,r^2} = \CF C_{D_8}(s)\). 
Why? Refer to the tree figure. We see that no bigger group may contain them. \\
\indent We may similarly determine the centralizers of \(D_{16}\), by referring the tree: \\

\section{composition series}
\subsection*{75. 2nd isomorphism thm.} (due to E. Noether) 
Let \(A, B \leq G\) and \(A \leq N_G(B)\)
Then \(AB \leq G\), and \(B \lhd AB\), and \(A \cap B \lhd A\), 
and above all there is isomorphism \(AB/B \cong A/A\cap B\). \par
\Ss{Proof.} The natural mapping \(\fG: g \Mp G/N\), when applied on \(H\), gives \(J:= \{hN: h \in H\}\), which \(\subseteq G/N\). 
It is a group since \(N\) is normal. 
The kernel is \(N\). \par
Now \(\fG(HN) =\) all of \(J\), while \(\fG(H)\) alone is also all of \(J\). 
Their kernel are, resp., \(N\) which \(\subseteq HN\), and \(H \cap N\) which \(\subseteq H\). 
Use 1st isomorphism theorem twice, so both of them is \(\cong \{hN: h \in H\}\). 

\subsection*{76. 3rd isomorphism thm.} (due to E. Noether) 
Suppose \(H, K \lhd G\), and \(H \leq K\). 
Then \(K/H \lhd G/H\) and \((G/H)/(K/H) \cong G/K\). \par
\Ss{Proof.} \(K/H \lhd G/H\), and I now show why. 
Introduce \(\fG: G/H \Mp G/K\) by \(\fG(gH) = gK\). 
To check it's well defined: consider \(gH = g'H\), then \(gk = ghh^{-1}k = g'h'h^{-1}k\), for some \(h'\). 
Since \(h'h^{-1}k \in K\), we see \(gK \subseteq g'K\), and similarly \(g'K \subseteq gK\). 
What's its kernel? It send \(gH\) to just \(K\), so \(g \in K\), and \(\Tw{ker}(\fG) = K/H\). 
It is normal in the domain \(G/H\). \par
Canonical homomorphism sends \(G\) to \(G/H\), forming cosets of \(H\). 
Another canonical homomorphism sends \(G/H\) to \((G/H)/(K/H)\) (we just checked normality), forming cosets of \(K/H\). 
The kernel of composition is of course \(K\). 
To see \((G/H)/(K/H) \cong G/K\), just apply 1st isomorphism thm. 

\subsection*{77. thm.} Consider \(A, B \leq G\), and \(N \lhd G\), and \(N \leq A\), and \(N \leq B\). Then: \par
(i) \(A \leq B \Eq A/N \leq B/N\). \par
(ii) \(A \leq B \Eq |B:A| = |B/N : A/N|\). \par
(iii) \(\Ab{A,B}/N = \Ab{A/N, B/N}\) and \(\subseteq G/N\). \par
(iv) \((A \cap B)/N = (A/N) \cap (B/N)\). \par
(v) \(A \lhd G \Eq A/N \lhd G/N\). \par
\Ss{Proof.} (i) is straightforward. 
(ii) follows from just finding group order. 
For (iii), notice \(aNbN = abNN = abN\), and this is true in any combination of \(a,b\). 
For (iv), that lhs \(\subseteq\) rhs is obvious; conversely suppose \(an = bn'\) for any \(a,b\), then, noting \(n \in A\), \(b\) must be in \(A\). 
For (v), the canonical map forming cosets preserves normality.

\subsection*{78. def.} \Ss{Composition series} of G is a sequence in which not only \(1 = N_0 \leq N_1 \leq N_2 \dotsc\leq N_k = G\), 
but also \(1 = N_0 \lhd N_1 \lhd N_2 \dotsb\lhd N_k = G\), 
and moreover \(N_{i+1} / N_i\) are simple

\subsection*{79. Jordan H\"older theorem.} [Nov. 2] If \(G\) has a composition series \EqGo{
1 \lhd N_0 \lhd \dotsb \lhd N_r = G \\
1 \lhd M_0 \lhd \dotsb \lhd M_r = G,
} then \(N_i\) and \(M_i\) are same sets, up to permutation and isomorphism. 
As a result, \(r\) is an invariant of \(G\). \par
\Ss{Proof.} This follows from refinement thm., which we study below. \par
[T.Y.J. --- for a simpler proof, see p.106 ex.9 \& 10. 
Namely, we first show the simplest case \(\{1\} \lhd \dotsb \lhd N_1 \lhd G\) and \(\{1\} \lhd M_1 \lhd G\). 
Because \(M_1 \leq N_G(N_1)\), by 2nd isom. thm., \(N_1 M_1 / M_1 \cong N_1 / N_1 \cap M_1\). 
But \(N_1\) is simple; 
so \(N_1 \cap M_1 = \{1\}\), 
and \(N_1 M_1 / M_1 \cong N_1\). 
For \(\Ev n_1,n_1'\), this implies \(n_1 m_1 M_1 n_1' m_1' M_1 = n_1 n_1' M_1\), 
or after a moment's thought, \(M_1 \lhd N_1\), 
which forces \(M_1 = N_1\), again by simplicity. \par
[If \(\{1\} \lhd \dotsb \lhd N_k \lhd G\) and \(\{1\} \lhd M_1 \lhd M_2 \lhd G\), where wlog., \(k \geq 2\), 
note that \(M_2 \cap N_k \lhd M_2 \lhd G\), while at the same time \(M_2 \cap N_k \lhd N_k \lhd G\). 
A moment's thought reveals that \(M_2 / M_2 \cap N_k\) must be simple: 
this fact follows from that conjugation orbits of \(M_1\) must coincide entirely several of its cosets or not, 
and that \(M_2 / M_1\) is assumed simple. 
Finally, one may use previous result to conclude \(N_k = M_2\), 
and by induction that all other \(N_i = M_i\).]

\subsection*{80. Butterfly lemma.} (Zassenhaus) This is a generalization of 2nd isom. thm. of Noether. \par
Let \(B \lhd A\) and \(D \lhd C\), all being sbgp.s of \(G\). 
Then \EqGo{
 \F{(A \cap C)B}{(A \cap D)B} \cong \F{(A \cap C)D}{(B \cap C)D}
}
\indent \Ss{Sketch of Proof.} The homomorphism \(\phi(x) = x(A \cap D)B\) maps \(A \cap C \Mp B(A \cap C)/B(A \cap D)\). 
By 1st isom. thm., and the fact that \(\Tw{ker}(\phi) = (A \cap D)(B \cap C)\), \EqGo{
 \F{A \cap C}{(A \cap D)(B \cap C)} \cong \F{(A \cap C)B}{(A \cap D)B}
} Similarly, repeat the argument with \(A\) and \(C\) interchanged, and \(B\) and \(D\) interchanged, \EqGo{
 \F{A \cap C}{(A \cap D)(B \cap C)} \cong \F{(A \cap C)D}{(B \cap C)D}
} \indent [T.Y.J. --- Introduce shorthand \(X := A \cap C\) and \(Y := A \cap D\). 
First check \(BY \lhd BX\). In fact, \(xbYB =xbBY =xBY =xYB =YxB =YBx =YBbx =BYbx\). 
The key is to note that \(BY = YB\) (since \(B \lhd A\)) and \(xY = Yx\) (since \(D \lhd C\)). 
Similarly, this being borne in mind, that \(\phi(x)\phi(x') = \phi(xx')\) is completely straightforward. \par
[Finally, \(\Tw{ker}(\phi)\) should be everything that lies in \(YB\), i.e., \(X \cap Y\). 
Let \(z= yb\), which is also \(\in X\). 
Since \(y \in X\), it suggests \(b \in X\), and hence \(z \in (A \cap D)(B \cap C)\).]

\subsection*{81. Schreier refinement thm.} Let \(G\) be a finite group, and \EqGo{
 \{1\} = H_0 \lhd H_1 \lhd\dotsb\lhd H_m =G. \\
 \{1\} = K_0 \lhd K_1 \lhd\dotsb\lhd K_n =G.
} are two normal series (not necessarily composition series, i.e., quotient being simple). 
Then in each case a new series must exist, which is a refinement of both, \EqGo{
 \{1\} = H_0' \lhd H_1' \lhd\dotsb\lhd H_{m'}' =G. \\
 \{1\} = K_0' \lhd K_1' \lhd\dotsb\lhd K_{n'}' =G.
} and furthermore \(\Ev i\), \(\Ex j\) s.t. \(H_{i+1}' / H_i' \cong K_{j+1}' / K_j'\). \par
\Ss{Proof.} Define \(G_{rm+s} = (G_{r+1} \cap H_s) G_r\), and similarly \(H_{rm+s} = (H_{r+1} \cap G_s) H_r\). 
It's easy matter to see they are each a new normal series, and moreover a refinement of the original one. 
Furthermore, by def., by Butterfly lemma and premises \(G_r \lhd G_{r+1}\) and \(H_s \lhd H{s+1}\), and finally by def. \EqGo{
 \F{ G_{um+v+1} }{ G_{um+v} }
 = \F{ (G_{u+1} \cap H_{v+1}) G_u }{ (G_{u+1} \cap H_v) G_u }
 \cong \F{ (H_{v+1} \cap G_{u+1}) H_v }{ (H_{v+1} \cap G_u) H_v }
 = \F{ H_{vn+u+1} }{ H_{vn+u} }.
}

\section{free groups}
\subsection*{82. def.} [Oct. 8] Consider \Ss{words} consisting of \Ss{alphabets}. 
On them, multiplication is defined as just composition. 
For example, \(aa \M bb =aabb\). 
A word is said \Ss{reduced}, if all possible cancellation are carried out. 
For example, \(acbb b^{-1}b^{-1} c^{-2}e =ac^{-1}e\).

\subsection*{83.} It's easy to explain (but perhaps laborious to write symbolically and rigorously), by induction, that the reduced form is unique. 
Words with the same reduced form comprise a equivalence class. 
And exactly the same way we composite words also becomes a well-defined operation of (now reduced) words. 

\subsection*{84.} It may be viewed as a group, and is called \Ss{free group}. 
All alphabets together generate it. 
Empty word is an identity. 
There is no relation between any elements. \par
For example, \(\BF Z\) under usual addition is infinite cyclic free group \(\Ab{g}\). 

\subsection*{85.} Suppose \(S\) generates a group \(G\), but the free group \(F(S)\) thus made basing on \(S\) invokes a natural, onto, homomorphism: \(g_1 g_2 \in \CF F(S)\) is mapped to \(g_1 \M g_2 \in G\). 

\subsection*{86.} Let \(G\) be a group, \(R \subseteq \CF F(S)\), then there exists a unique smallest normal sbgp.\ \(N_R \lhd G\), with \(N_R \supseteq R\), which is \(\{g \Ab{R} g^{-1}\}\). \par
With \(S\), \(R\) given, can the word game be proved not to collapse, that is to say every word is the identity? 
This is proved, by Alan Turing, to be impossible. 

\subsection*{87.} In dihedral group, relations \(r^n = 1, s^2 = 1, rsrs = 1\) sufficiently defines it. 
Now we define a group G in general, by a set of relations \(R\), and symbols \(S\). 
Here, \(r \in R\) is in the form \(\cdot=1, \dotsc \cdot=1\), so \(N_R\) should be 1. \par
If that be the case, under homomorphism \(h: \CF F(S) \Mp G\), and \(N_R = \Tw{ker}(h)\), and, therefore by 1st homomorphism thm., that \(\CF F(S) / N_R \cong G\). 

\section{solvability}
\subsection*{88.} Another fundamental problem : given groups \(G_1, G_2\), both finitely presented, can one write down non-trivial homomorphisms? \par
We have only to check that the relation hold for generators. For example, to check that \(D_6 \Mp D_{12}\) can be done, 
we assign \(s \Mp s'\) and \(r \Mp r'\), 
and check: \EqGo{
 \fG( s^2 ) = 1, \fG(r^3) = 1, \fG(rsrs) =1
}

\subsection*{89.} A set \(S\) is given, which have free group \(\CF F(S)\). Question: is any sbgp.\ \(S'\) of free group a free group? 
The answer is yes, known as \Ss{Schreier's thm}. 
But the generating set of \(S'\) may come with larger cardinality than that of \(S\).

\subsection*{90.} \(G\) is said \Ss{solvable}, if it has a series of sbgp.s \EqGo{
 \{1\} = G_0 \lhd G_1 \lhd \dotsb \lhd G_s = G, 
} where \(G_{i+1}/G_i\) is abelian. 
For prime number \(p_i\), they is thus \(\cong \BF Z_{p_i}\) 
\Ss{Simple groups} are not solvable, i.e. without proper normal sbgp.s. 
For example, \(\BF Z_p\) are simple, and so are alternating group \(A_n\). 

\subsection*{91.} Suppose \(|G| = p^3\), where \(p\) is odd prime and \(x^p = 1,\; \Ev x \in G\). 
If \(G\) is furthermore non-abelian, then G must \(\cong \CF H(\BF F_p)\) (left to the reader). 

\subsection*{92.} Recall that by class equation, any \(p\)-group \(G\) has non-trivial center \(\SF Z(G)\). 
ex: \(|\SF Z(\CF H(\BF F_p))| = p\). \(|\SF Z(Q_8)| =2\). \(\SF Z(D_8)| =2\). 

\subsection*{93. examples.} Recall if prime number \(p\) divides \(|G|\) and \(|G:H| = p\), then \(H \lhd G\). 
For example, \(Q_8\) has sbgp.s \(\Ab{\Tw i}, \Ab{\Tw j}, \Ab{\Tw k}\), each of index 4. 
In general, there may've been many sbgp.s of index p inside a p group; they do not conjugate to each other. 

\subsection*{94. Theorem.} [p.188 thm.1(iv)] \(|P|=p^a\). 
If \(H < P\) strictly, then \(H < N_P(H)\) strictly. \par
\Ss{Proof.} If \(P\) is abelian, it's of course normal. 
Otherwise suppose it's non-abelian. 
If \(H < P\) strictly, as assumed, we have once noted, from class equation, that \(\SF Z(G)\) is nontrivial. 
If \(\SF Z(G) - H \neq \varnothing\), the sbgp.\ generated from \(\SF Z(G)\) and itself (\(H\)) would \(\rhd H\). 
Let \(\SF Z(G) \leq H\). 
Then, too, \(\SF Z(G) \lhd P\). 
\(P/ \SF Z(G)\) is of smaller order, since \(\SF Z(G)\) is nontrivial. 
Thus motivated, we may well have used induction in advance w.r.t. \(|P|\). 
Owing to assumption, a certain \(H'/\SF Z(G) < H/\SF Z(G)\). 
Finally \(H' < H\). 

\subsection*{95.} [p.188 thm.1(v)] \(|P|=p^a\). Every maximal normal sbgp, say \(M\), of \(P\) has index \(p\) and normal. \par
\Ss{Proof.} For proper, maximal \(M\), it follows from item 94 that \(M < N_P(M)\). 
Now \(N_P(M)\), being a larger sbgp, is then all of \(P\). 
So \(P/M\) exist, with natural operation endowed. 
It cannot possess proper sbgp.\ (otherwise \(M\) is not maximal). 
But \(|M|=\) some \(p^b\), and in part (iii) we saw it has sbgp.s of all powers of \(p\). 
The only possibility is that \(|P/M| =p\). 
Suppose \(P\) is non-abelian. 
\(H < P\) by hypothesis. 

\subsection*{96.} [p.188 thm.1(ii)] \(|P|=p^a\) and \(1< H \lhd P\). 
Then \(H \cap \SF Z(P) \neq 1\). 
In particular, every normal sbgp.\ of order \(p\) lies inside \(\SF Z(P)\). \par
\Ss{Proof.} From class equation, where \(\CF C\) denotes centralizer, \EqGo{
 |H|= |H \cap \SF Z(P)| + \sum_i \F{|P|}{|\CF C(h_i)|}.
} Here, every element in \(H \cap \SF Z(P)\) is, by commutativity, a trivial orbit, 
and every non-trivial orbit of \(H\) (which is a sbgp.\ of \(p\)-group \(P\)) is multiple of \(p\). 
So \(|H \cap \SF Z(P)|\) is multiple of \(p\), 
and hence is nontrivial. 
Especially, if \(|H|\) is so small as \(=p\), then the nonempty \(|H \cap \SF Z(P)|\) must be all of \(H\). 

\subsection*{97.} [p.188 thm.1(iii)] \(|P|=p^a\) and \(H \lhd P\). 
Then \(\Ev p^b\) that divides\(|H|\), there is a normal sbgp.\ of \(H\) that has order \(p^b\). 
Note that in particular, \(P\) has normal sbgp.\ of order \(\Ev p^b\). \par
\Ss{Proof.} By item 96, \(H \cap \SF Z(P) \neq \{1\}\). 
It, being sbgp.\ of \(P\), has order at least \(p\). 
Divide by it, and use induction; 
by virtue of canonical isomorphism that sends set to coset, the result follows. 

\subsection*{98. def.} Let \(Z_1(G) = \SF Z(G)\), the center of \(G\). 
We form \(\SF Z(G/Z_1) \Mp Z_2\) by natural isomorphism, and so forth. 
Clearly, \(Z_1 \lhd G\), so are \(Z_i\) that follows. 
The \Ss{upper central series} consists of \(1= Z_0(G) \leq Z_1(G) \leq  Z_1(G) \leq\dotsc\). 

\section{finite abelian groups}
\subsection*{99.} \(G\) is called a \Ss{nilpotent group} if \(Z_l(G) = G\) for some \(l \in N\). 
The smallest \(l\) is called the \Ss{nilpotent class} of \(G\). \par

\subsection*{100.} Observe: (i) All \(Z_l(G)\) are characteristic sbgp.s of \(G\). 
[T.Y.J. --- Indeed, if \(Z_l\) is so, the same is \(G/Z_l\), and its center \(\SF Z(G/Z_l)\), which is \(Z_{l+1} / Z_l\), 
and by canonical isomorphism, so (characteristic) is \(Z_{l+1}\). 
Starting with the identity, in this manner go all the way to \(G\).] 
In particular, when applied to conjugacy action, it's seen \(Z_j(G) \lhd G,\; \Ev j\). 
In this regard, when \(Z_{l+1}(G) / Z_l(G)\) are abelian groups, the nilpotent groups, by def., are solvable groups. 

\subsection*{101.} Dihedral group \(D_{2^n},\; n=3,4,5\dotsc\), is nilpotent of class \(n-1\) --- it's left to the reader. 
If \(n\) is not power of 2, \(D_{2n}\) is not nilpotent. 
These are solvable groups, but not nilpotent. 
(Because dividing by its center gives cyclic \(\BF Z_n\)). \par

\subsection*{102. thm.} Let \(|P| = p^a\) where \(a = 2,3,\dotsc\). 
Then \(P\) is nilpotent of class \(\leq a-1\). \par
\Ss{Proof.} For \(i= 0,1,\dotsc\), \(|P/Z_i(P)| >1\) which implies \(\SF Z(P/Z_i(P) \neq 1\). 
This, also being a \(p\)-group, has non-trivial center \(:= Z_{i+1}(P)/Z_i(P)\), 
Here \(|Z_{i+1}(P)| \geq p|Z_i(P)| \geq p^{i+1}\). 
Continue the process, until finally \(|Z_a(P)| \leq p^a\), which is just \(P\). 
But \(P\) cannot have class \(a\). 
Why so? When we arrived \(Z_{i-2}(P)\) which has index \(p^2\) in \(P\), 
we see \(P/Z_{i-2}(P)\) to be abelian, and its center is all of it. 
Thus \(Z_{i-1}(P) =P\). 
Since \(i-2 \leq a-2\), \(i-1 \leq a-1\). 

\subsection*{103. thm.} Let \(G\) be a finite group, \(p_i\)'s are distinct primes, \(i=1,\dotsc,s\), that each divides \(|G|\). 
Let \(P_i \in \Tw{syl}_{p_i}(G)\) be one of the Sylow \(p_i\)-sbgp.s. 
Then all of these are equivalent: 
(i) \(G\) is nilpotent. 
(ii) If \(H < G\), then \(H < N_G(H)\). 
(iii) \(P_i \lhd G\) for all \(i\), i.e., Sylow sbgp.s are all normal. 
(iv) \(G \cong P_1 \X\dotsb\X P_s\). \par
\Ss{Proof.} To show (i)\(\Ip\)(ii). Note that \(G/\SF Z(G)\) is nilpotent as \(G\) is nilpotent --- 
indeed, we just divide, by \(\SF Z(G)\), each of the upper central series. 
Then we argue inductively and assume \(H/\SF Z(G) < G/\SF Z(G) \Ip H/\SF Z(G) < N_{G/\SF Z(G)}(H/\SF Z(G))\). 
By canonical isomorphism, \(H < N_G(H)\). \par
To show (ii)\(\Ip\)(iii). 
Consider some Sylow sbgp.\ \(P_i\). 
Sylow thm. says it \(\lhd G\). 
Textbook Corollary 4.20 says it's characteristic in \(N_G(P)\). 
But \(N_G(N_G(P)) > N_G(P)\), and since conjugacy maps the characteristic \(P\) to itself, and also \(N_G(P)\) to itself, 
we have \(P \lhd N_G(P)\), 
and hence \(N_G(P) < N_G(N_G(P))\) (by def. of \(N_G(P)\)). 
In conclusion, \(N_G(N_G(P)) = N_G(P)\). 

\subsection*{104.} [Oct. 19] Every finite \(p\)-group, \(A\), is a product of cyclic groups. \par
\Ss{Proof.} Consider elementary abelian \(p\)-group (i.e. \(\Ev g \in A,\; g^p = 1\)). 
Then \(A \cong \Ab{x} \X M\), and we show inductively by inspecting instead on \(M\). 
Find maximal \(M \leq A\). 
If \(|M:A| \neq p\), which must < p
consider \(x' \in A' := A/M\). 
There exists \(y' \in A', y' \in A' - \Ab{x}\). 
Because induction assumption and \(A'\) has smaller order than \(A\), so \(y'\) has order \(p'\). 
Then generate \(\Ab{y'}\); its preimage \(\leq A\) has order at least \(p|M| > |M|\). 
This is a contradiction. \par
Consider general \(A\), and induction in terms of \(|A|\). 
Consider homomorphism \(\fG: A \Mp A,\; \fG(x) = x^p\). 
\(K = \Tw{ker}(\fG)\) is, by def., elementary abelian. 
So is \(A/H\), where \(H = \Tw{img}(\fG)\). 
Indeed, every \(a\) when taken \(p\)-th power becomes inside \(H\). 
By above, \(H = \Ab{h_1} \X\dotsb\X \Ab{h_r}\). 
\par
If \(K \subseteq H\), then \(K \cap H = K\), 
then \(H \cap K\) is also product of cyclic group: \(\Ab{h_1}^{s_1} \X \Ab{h_r}^{s_r}\) where \(h_i^{s_i p} = 1\). 
And \(\Ab{g_1} \X\dotsb\X \Ab{g_r}\) (where \(\fG(g_i) = h_i\), every preimage exists) has order same as \(A\), by isomorphic theorem. \par
If \(K - H \neq \varnothing\), find \(x\) in it. 
To apply the result in the beginning of the proof, \(A/H\) is the product of some \(M\) and \(\Ab{x}\). 
By induction assumption, \(M\) is product of cyclic groups. 

\subsection*{105.} Every finite abelian group, \(A\), is a product of cyclic groups. \par
\Ss{Proof.} Because abelian, divide by Sylow sbgp, and so on; 
it becomes direct product of Sylow sbgp.s. 
But we just saw they are direct product of cyclic groups. \par
Note that, accordingly, abelian groups, up to isomorphism, corresponds to partition (several numbers whose sum is some number, that is the order). 

\subsection*{106.} One may check \(\BF Z_4 \rtimes \BF Z_4\) homomorphic to quaternions \(Q_8\). 
Here \(\fG_a(q) = q^{-1}\), where \(a\) is the generator of (the latter) \(\BF Z_4\), and \(\Ev q \in Q_8\). 
Please verify all rules in \(Q_8\) are valid. 

\subsection*{107. Group determinants.} [Nov. 2] The \Ss{group determinant}, \(\tG(G;x_g)\), of \(G\) is, informally, the determinant consisting of the multiplication table with each entry \(g\) being replaced by an free variable \(x_g\) (below for an example). 
Also recall [see text. p.167] that, for finite abelian group \(G\), its \Ss{dual}, \(\hat{G}\), is defined as the group of all homomorphism from \(G\) into the group generated by roots of unity in \(\BF{C}\). \par
\Ss{Dedekind's theorem} states \EqGo{
 \tG(G;x_g) = \prod_{\chi \in \hat{G}} \sum_{g \in G} \chi(g) x_g
} This way, the study of group \(G\) done on its det and is broken up into factors. 

\subsection*{108.} A special case is that of a cyclic group, say \(H = \Ab{x_0}\), where \(|H|=N\), which the reader may verify explicitly. It reads (\(\xi = \ee^{2\pi \ii/N}\) \begin{gather*}
 \det \begin{bmatrix}
  x_0, &x_1, &x_2, &\cdots, &x_{N-1} \\
  x_1, &x_2, &x_3, &\cdots, &x_0 \\
  x_2, &x_3, &x_4, &\cdots, &x_1 \\
  \vdots, &\vdots, &\vdots, &\ddots, &\vdots \\
  x_{N-1}, &x_0, &x_1, &\cdots, &x_{N-2}
 \end{bmatrix}
 = \prod_{j=0}^{N-1} \sum_{k=0}^{N-1} \xi^{jk} x_k
\end{gather*} The special case for \(N=3\), (\(\oG = \ee^{2\pi \ii/3}\) \begin{gather*}
 \det \begin{bmatrix}
  x_0, &x_1, &x_2 \\
  x_1, &x_2, &x_0 \\
  x_2, &x_0, &x_1
 \end{bmatrix}
 = (x_0 +x_1 +x_2) (x_0 +\oG x_1 +\oG^2 x_2) (x_0 +\oG^2 x_1 +\oG x_2)
\end{gather*} To show this, Vandemonde determinant formula is used (see p.619). 

\section{more on simplicity}
\subsection*{109.} [Nov. 9] Simple group \(G\) satisfies (i) \(G = [G,G]\), since a derived group is normal [see text sec. 5.4]. 
(ii) \(G\) acts on S non-trivially, \(G \Mp S_{|S|}\) which = all permutations of \(S\). 
[Why? consider cosets of \(H < G\). It induces a map that permutes cosets. But it has no kernel, which is normal.]

\subsection*{110.} \(A_5\) is simple. It has no centre, so by counting conjugacy classes, actually \(60 = 1 + 12 + 12 + 15 + 20\) (though it's not easy task to find them). 

\subsection*{111. Simplicity of \(A_n\).} To show \(A_n\) is simple, \(n \geq 5\). 
Use math induction. 
Stabilizer of \(i\) is \(G_i\), permutations fixing \(i\), is just \(= A_{n-1}\). 
If \(H \lhd A_n\), then clearly \(H \cap G_i \lhd G_i\). 
Either \(H \cap G_i =1\), or \(H \cap G_i = G_i\), because \(A_{n-1}\) is simple. \par
Recall identity \(\sG \tG \sG^{-1} = \tG_{\sG(i)}\) (\(\tG\) permutes instead on indices permuted by \(\sG\)). 
If any \(G_i \leq H\), then all \(G_j \leq H\), and (clearly) \(H\) is just all of \(A_n\). 
Otherwise, no non-identity fixes any elements. 
Now, \(\tG'(i)=\tG(i)\) indicates \(\tG'\tG^{-1}(i)=i\) for some \(i\). 
This element, also being in \(H\), must \(=e\). 
So no two permutations move any index to the same one. \par
Again use \(\sG \tG \sG^{-1} = \tG_{\sG(i)}\), and it's easy to construct elements, whatever their composition cycle is, that keep fixed some of them. For example, \EqGo{
 (1,2,3)(4,5) \Mp (1,2,4)(5,3)
} Then both 1 and 2 are fixed. See text p.150 for details. 

\subsection*{112.} A \Ss{block} \(B\) is either invariant or mapped to something disjoint with it. 
That is, \(g(B)=B\) or \(g(B) \cap B = \varnothing\). 
So the way group \(G\) acts on \(B\) is homomorphic to \(S_n\), if there are \(n\) blocks. 

\subsection*{113.} \(G\) acts \Ss{doubly transitive}, if \(\Ev G_a\) is transitive (there is only one orbit). 
It easy to show, doubly transitive \(G\) forces there to be no block, or \Ss{primitive}. 

\subsection*{114. examples.} in vector space \(F^n\) (\(F\) is a field), there is an (clearly) equivalence relation \(b \sim a \Eq \V{b} = \aG \V{a}\), for some \(\aG \neq 0\). 
The projection on \(F^{n-1}\), which is \(P^{n-1}(F)\), parametrize 1-d space \(F\). 
There is natural action of \(\CF{SL}_n(F)\) (isometric matrices) on \(P^{n-1}(F)\). \par
Then it is always possible to find a matrix that fixes \(n\)-th axis, but permutes \(y_1\) to be \(y_2\), for \(\Ev y_1, y_2 \in P^{n-1}(F)\) [it just determines a basis transformation]. 
In other words, \(\CF{SL}_n(F)\) is doubly transitive on \(P^{n-1}(F)\). 

\subsection*{115. Iwasawa lemma.} Let \(G\) be a perfect group (\(=[G,G]\)) that acts transitively, primitively, and non-trivially on \(S\). 
Suppose at \(x \in S\), the stabilizer \(G_x\) has normal abelian sbgp.\ \(A\) whose conjugates generate \(G\). 
Then \(G\) must be simple. \par
\Ss{Proof.} Suppose \(\{1\} \neq H \lhd G\). 
Consider stabilizer \(G_x\). 
Now, \(H - G_x \neq \varnothing\). [otherwise \(H \subseteq G_x\), that's to say normal \(H\) fixes some \(x \in S\). This cannot be: \(G\) is transitive, so faithful, and if \(g(y) \neq y\), then \((ghg^{-1})y \neq h(y)\).] 
By exercise (4.1.7d), \(G_x\) is maximal (no sbgp.\ containing it). 
So, \(G_x H\) is all of \(G\). \par
Claim \(A \lhd AH\): indeed, choose some element \(y\) of \(AH\) and find the form \(y= g_x h\) by just above, 
and check \(h^{-1} g_x^{-1} A g_x h = h^{-1} A h = Ah'h^{-1} \subseteq HA\), by \(A\) being normal. 
Also, by assumption \(gAg^{-1}\) union to be \(G\), 
it follows \(G \subseteq AH\), 
and since of course \(AH \subseteq G\), in fact \(G = AH\). 
So \(A \lhd G\). \par
Use 3rd isomorphism thm. to get \(G/H \cong AH/H \cong A / A \cap H\). 
Now \(A\) is abelian and hence \(A / A \cap H\) is abelian, and so is \(G/H\). 
By property of \([G,G]\) (see sec.5.4, pro.7(4)), we have \(H \geq [G,G]\), which (being perfect) \(= G\). 
Of course \(H \leq G\). 
This forces \(H=G\). 
So \(G\) has no proper normal sbgp. 

\subsection*{116.} Application to \(\CF{PSL}_n(F) := \CF{SL}_n(F) / \SF{Z}(\CF{SL}_n(F))\). 
Let \begin{gather}
 a= \begin{bmatrix}
  1, &0, &0, &\cdots, &0 \\
  a_1, &1, &0, &\cdots, &0 \\
  a_2, &0, &1, &\ddots, &\vdots \\
  \vdots, &\vdots, &\ddots, &1, &0 \\
  a_n, &0, &\cdots, &0, &0 \\
 \end{bmatrix}
\end{gather} and \(A\) be the collection of such \(a\). 
Easy to show, it \(\cong F_n\), and is abelian. 
And \begin{gather}
 g= \begin{bmatrix}
  1, &0 \\
  \V{v}, &g'
 \end{bmatrix}
\end{gather} where \(\det{g'}=1\), to write in a block matrix --- these comprise stabilizer of \(\hat{\V{e}}_1\), as can be verified. 
Similar one may find stabilizer for other \(\hat{\V{e}}_i\). 
Also one may check \(A \lhd G_x\). \par
The key is whether \(A\) has conjugate orbits cover all of \(G\). 
To quote the result, \(\CF{PSL}_n(F)\) is simple, whenever \(n \leq 3\), or for \(n=2\), whenever \(|F| >3\). 

\section{miscellaneous examples}
\subsection*{117. generalized quaternion.} [Oct. 26] (p.178, example 3) Let us define \(Q_{2^n} := Z_{2^{n-1}} \rtimes Z_4 / \Ab{(2^{n-2},2)}\), with automorphism defined by \((a,b)(c,d) = (a+(-1)^b c, b+d)\). 
Clearly \(\Ab{2^{n-2},2} \in Q_{2^n}\) has order 2  
So \(|Q_{2^n}| = 2^{n-1} \M 4 /2 = 2^n\). \par
In other words: \(Q_{2^n} = \Ab{ x,y: x^{2^{n-1}} =1 =y^4, x^{2^{n-2}} =y^2, yxy^{-1} =x^{-1} }\). 
The 1st relation comes from \(Z_{2^{n-1}}\) and \(Z_4\). 
The 2nd comes from dividing \(\Ab{(2^{n-2},2)}\). 
The 3rd from our def. of automorphism: that maps \(x\) to \(-x\). 
It has generator \(x := (1,0) + \Ab{2^{n-2},2},\; y := (0,1) + \Ab{2^{n-2},2}\). 
\(\Ab{x}\) is an sbgp.\ of index 2 inside \(Q_{2^n}\). 
Every element not in \(\Ab{x}\) has order 4 (as \(yxy^{-1} =x^{-1}\)). \par
In particular, please check \(Z4 \rtimes Z4 /\Ab{(2,2)} \cong Q_8\). 

\subsection*{118.} Back to linear algebra. 
\(V\) being finite-dimensional vector space over \(F\). 
Given, on \(V\), a symmetric bilinear form, \(B: V \X V \mapsto F\), which is denote as \(B(u,v) := \Ab{u,v}\), 
and such that: \(B(u,v) =0,\; \Ev v \Ip u=0\). \par
It is a generalization of usual inner product. 

\subsection*{119.} Collection of transformations \(A\) into \Ss{orthogonal bases} forms a group, denoted \(\CF{O}(n,\BF R)\). 
Recall they satisfy \(A^\Tw{t} A = I\), hence \(\Tw{det} A = \pm 1\). 
Pick up those for which \(\Tw{det} A = 1\); it's denoted as \(\CF{SO}(n,\BF R)\).

\subsection*{120. Platonic solids.} (for detailed discussion, see M. Artin's \textit{Algebra}) 
Indeed, for every group object inherit with some symmetry, action on it forms a group. 
We've seen cyclic groups, dihedral groups, and now we may well define tetrahedral, cubic, octahedral, dodecahedral, and icosahedral rotation groups. \par
Consider \(G \subseteq \CF{SO}_3, |G| < \oo\), acting on a polyhedron. 
A set of point is said \Ss{poles} of \(g\), whenever they are fixed point. 
Because we discuss \(\CF{SO}_3\), there are always 2 poles only. \par
Let \(P\) be the set of all poles corresponding to all elements \(g \in G\), possibly repeated. 
\(G\) acts on \(P\). 
If \(p\) is a pole, then \(g(p)\) too is a pole. 
Consider stabilizer \(G_P(g)\) and disjoint orbitals \(|O_i|\)'s of \(hgh^{-1}\) acting on poles, where \(g,h \in G\). 
For simplicity, \(r_i := |G_P(g_i)|\), and \(s_i := |\CF O_i|\). \par
All poles are as many as \(2(|G|-1)\). 
And the number is also \(\sum_i s_i (r_i -1)\), where  and \(r_i\)'s are number of stabilizers. 
It just accounts the fact that all poles (corresponding to every non-identity action) comprise the union of orbits, 
and is counted \(r_i -1\) times by different \(g\) action that leaves it invariant. 
Meanwhile, recall \(s_i r_i =|G|\). 
Thus, \EqGo{
 \sum_i s_i (r_i -1) = 2(|G| -1). 
 \Ip \sum_i \Rb{ 1- \F{1}{r_i} } = 2 - \F{2}{|G|}. 
} \indent If there be only 1 orbit, \(1 - 1/r_1 = 2 - 2/|G|\), which is impossible. \par
If there be 2 orbits, \(1 - 1/r_1 + 1 - 1/r_2 = 2 - 2/|G|\). 
Only possibility: \(r_1 = r_2 = |G|\), 
so \(G\) is cyclic. \par
Consider case that there are 3 orbits. 
Possibilities are \EqGo{
 (r_1, r_2, r_3) = (2,3,3),\; (s_1, s_2, s_3) = (6,4,4),\; |G|=12; \\
 (r_1, r_2, r_3) = (2,3,4),\; (s_1, s_2, s_3) = (12,8,6),\; |G|=24; \\
 (r_1, r_2, r_3) = (2,3,5),\; (s_1, s_2, s_3) = (30,20,12),\; |G|=60;
} In the 1st case: orbit \(O_3\) is faces of tetrahedron (3 edges for every triangle), 
or (may well be seen as) vertices of tetrahedron (4 vertices). 
In the 2nd case: orbit \(O_3\) is faces of a cube (6 edges for every square), 
or vertices of octahedron (6 vertices). 
In the 3rd case: orbit \(O_3\) is faces of a dodecahedron (5 edges for every pentagon), 
or vertices of icosahedron (12 vertices).

\subsection*{121. homotopy.} To define group law on homotopy classes of closed curves starting and ending at the same point. [see Ahlfors, \textit{Complex Analysis}, chap.8, sec.1.5, p.292 to 294]. \par
Put precisely, let \(\gG_1, \gG_2 : [a,b] \Mp X\) be two closed curves in a topological space, \(X\). 
We say \(\gG_1\) is \Ss{homotopic to} \(\gG_2\), when \(\gG_1(a) =\gG_2(a) =\gG_2(b) =\gG_1(b) =z_0\in X\), 
and there exists continuous function \(H\) from \([a, b]\times[0, 1]\) to \(X\), for which: \(H(t,0) =\gG_1(t)\), \(H(t,1) =\gG_2(t)\) for \(\Ev t\in [a,b]\) and \(H(a,u) =z_0 =H(b,u)\) for \(\Ev u \in [0,1]\). \par
Given two closed curves \(\gG_1\) and \(\gG_2\) in \(X\) with the same initial and terminal point \(z_0\), define the composition law \((\gG_1, \gG_2) \Mp \gG_1\gG_2\). 
Here \(\gG_1\gG_2\) is the curve obtained from tracing first \(\gG_1\) then followed by \(\gG_2\), as a closed curve defined on the interval \([a, 2b -a]\) which is associated with interval \([a, b]\) by means of a strictly increasing function. \par
Verify this is a group law on homotopic classes of closed curves. \\
\indent 1. Check it is well-defined, if \(\gG_1\) is homotopic to \(\beta_1\), and \(\gG_2\) is homotopic to \(\bG_2\), then \(\gG_1 \gG_2\) is homotopic to \(\bG_1\bG_2\). \\
\indent 2. Check the associativity. \\
\indent 3. What is the identity for this composition law? \\
\indent 4. Given closed curve \(\gG\) on \(X\), write down a closed curve whose homotopy class is the inverse of the class represented by \(\gG\). \par
This group of homotopy classes is called the \Ss{fundamental group of} \(X\) \Ss{with base point} \(z_0\). 
It is an (algebraic) invariant of the point topology space \((X, z_0)\). \par
\indent [T.-Y. J. --- This is fairly trivial, once one grasps the main idea. 
[Easy to verify being homotopy is an equivalence relation. 
Trivial \(H\) connects \(\gG\) to itself. 
Reverted \(H\) connects \(\gG\) to \(\bG\), whenever \(H\) connects \(\bG\) to \(\gG\). 
If \(\aG\) and \(\bG\) are homotopic, and too are \(\bG\) and \(\gG\), then concatenate two \(H\) in parameter \(u\) to see \(\aG\) and \(\gG\) are homotopic. \par
[Now, concatenation of \(\gG_1\) and \(\gG_2\) clearly may be mapped, by virtue of a map concatenated in parameter \(t\), to concatenation of \(\bG_1\) and \(\bG_2\); of course, a shrink of domain is needful, so as to fit \([0,1]\).  
Choose representatives \(\aG\), \(\bG\) and \(\gG\); then, concatenation in parameter \(t\) is obviously associative. \par
[Identity is the (trivial) class consisting solely of constant function that remains always at \(z_0\). \par
[Finally, consider a curve \(\gG\), and another one \(\tilde\gG\) with same path but opposite sense of rotation. 
Then \(\tilde\gG\) represents a class that consists of inverse of \(\gG\), for their multiplication is the identity.]

\end{document} 
%%%%%%%%%%%%%%%%%%%%%%%%%%%%%%%%%%%%%%%%%%%%%%%%%%%%%%%%%%%%
%~%~%~%~%~%~%~%~%~%~%~%~%~%~%~%~%~%~%~%~%~%~%~%~%~%~%~%~%~%~
Yay~below this line nothing is printed.
