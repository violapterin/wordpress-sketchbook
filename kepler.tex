\documentclass{article}
\usepackage[utf8]{inputenc}
\usepackage{amsmath}

\title{Kepler problem}
\author{Tzuyu Jeng}
\date{Nov 15, 2018}

\newcommand{\F}{\frac}
\renewcommand{\S}{\sqrt}
\newcommand{\Gb}{\beta}
\newcommand{\Gg}{\gamma}
\newcommand{\Gq}{\vartheta}
\newcommand{\La}{\langle}
\newcommand{\Ra}{\rangle}
\newcommand{\Bd}{\cdot}
\newcommand{\Ed}{\equiv}
\newcommand{\Xqd}{\dot{\vartheta}}
\newcommand{\Xqdd}{\ddot{\vartheta}}
\newcommand{\Xs}{\sin \vartheta}
\newcommand{\Xc}{\cos \vartheta}
\newcommand{\Xss}{\sin^2 \vartheta}
\newcommand{\Xcc}{\cos^2 \vartheta}
\newcommand{\Tct}{\;\mathrm{const}\;}

\begin{document}

\maketitle

We derive Newton's inverse square of distance law for gravitation, given Kepler's three laws.
Whenever possible, we avoid using sophisticated vector analysis properties, and stick to basic calculus.

Kepler's first law states the orbit, by which a planet revolves around the Sun, is an ellipse.
We know the ellipse is confined within a plane, by Newton's third law.
Without loss of generality, let the length be scaled so that the longer axis of ellipse along the \(x\) axis has length \(1\).
And let the shorter axis of ellipse along the \(y\) axis has length \(\beta <1\), so that the focal point lies on the x-axis.
The sun lies at \(\La \Gg, 0 \Ra\) the focal point of ellipse.
Introduce the focal distance
\begin{align}
\Gg \Ed \S{1 - \Gb^2}
\end{align}
Define the position vector \(r\), velocity \(v\), acceleration \(a\).
Vectors \(r\), \(v\), \(a\) are understood as a function of \(\Gq\), the angle originating from the sun, and \(\Gq= \Gq[t]\) is a function of time \(t\).
\begin{align}
r = &\La \Xc - \Gg, \Gb \Xs \Ra \\
v = &\La - \Xqd \Xs, \Gb \Xc \Ra \\
a = &\La -\Xqdd \Xs -\Xqd ^2 \Xc,
\Gb \Xqdd \Xc -\Gb \Xqd^2 \Xs \Ra
\end{align}
% % % % % % % % % % % % % % % % % % % % % % % % % % % % % % % % 
Meanwhile, Kepler's second law states that the area of the triangle swept by the line that passes through the planet and the sun, in unit time, is constant.
To express that triangle's area, we invoke a
vector identity below that gives the area spanned by two vectors.
For simplicity, we actually consider the square of the area.
\begin{align}
\| r \| ^2 \| v \| ^2 - (r \Bd v) ^2 = \Tct
\end{align}
% % % % % % % % % % % % % % % % % % % % % % % % % % % % % % % % 
The expressions for \(r\) and \(v\) yield a constraint on \(\Xqd\), namely,
\begin{align}
&(\Xcc - 2 \Gg \Xc + \Gg^2 + \Gb^2 \Xs^2)
\Bd \Xqd (\Xss + \Gb^2 \Xcc) \\
&\quad - \Xqd^2 (-\Xc \Xs + \Gg \Xs + \Gb^2 \Xs)^2 \\
= &\Tct
\end{align}
By arranging, by \(\Xss + \Xcc = 1\), by absorbing \(\Gb^2\) into the constant, and by pulling out \(\Xqd^2\),
\begin{align}
\Xqd^2 \Bd (1 - \Gg \Xc)^2
= &\Tct
\end{align}
Here \(1 - \Gg \Xc >0\), and if we agree \(\Gq\) revolves counterclockwise, \(\Xqd >0\) too.
\begin{align}
\Xqd \Bd (1 - \Gg \Xc)
= &\Tct = K
\end{align}
Differentiation on both sides gives
\begin{align}
\label{constraint-angle-derivative}
\Xqdd (1 - \Gg \Xc) +\Xqd^2 \Gg \Xs
= 0
\end{align}
Still after some manipulation, we have, except perhaps at the point
where both sides are singular,
\begin{align}
\frac {-\Xqdd \Xs -\Xqd^2 \Xc} {\Xc - \Gg}
=\frac {\Xqdd \Xc -\Xqd^2 \Xs} {\Xs}
\end{align}
This indicates that the force exerted by the Sun is along the same line 
of the Sun and the planet, but opposite direction.
% % % % % % % % % % % % % % % % % % % % % % % % % % % % % % % % 
From previous expression of \(a\),
\begin{align}
&\| r \| ^4 \| a \| ^2 \\
=&(1 - \Gg \Xc)^4
\Big(
\Xqdd^2 (\Gb^2 \Xcc + \Xss) \\
&\quad + 2 \Xqdd \Xqd^2 \Gg \Xc \Xs
+ \Xqd^4 (\Xcc + \Gb^2 \Xss)
\Big)
\end{align}
It may be hard to believe, but by \(\Xss + \Xcc = 1\), \(\Gb^2 + \Gg^2 = 1\), and repeatedly using \eqref{constraint-angle-derivative},
\begin{align}
&\| r \| ^4 \| a \| ^2 \\
=&(1 - \Gg \Xc)^2 \Bd \Xqd^4 \Gg^2 \Xss \Bd (\Gb^2 \Xcc + \Xss) \\
&\quad +(1 - \Gg \Xc)^3 \Bd \Xqd^2 (-\Gg \Xs) \Bd 2 \Xqd^2 \Gg^2 \Xc \Xs \\
+&(1 - \Gg \Xc)^2 \Bd \Xqd^4 \Bd (\Xcc + \Gb^2 \Xss) \\
=&K^2 \frac {(1 - \Gg \Xc)^2} {(1 - \Gg \Xc)^2} = K^2
\end{align}

\end{document}
